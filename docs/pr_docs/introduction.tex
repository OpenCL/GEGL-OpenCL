
For high color resolution film work being able to edit frames in a paint
package with minimal color data loss is extremely important. If a single frame
from a scene must be edited in a package whose format doesnt match the native
format of the image, it can mean the entire scene must be opened and saved in
that paint package to ensure consistent color throughout.   

The differences in dynamic range and white point choice for image formats cause
problems like this and many others in the paint and image-editing world of film.
Each studio or program has its own version of 16bits/channel, and finding a
paint package that can handle a particular type and can be installed on every
machine is difficult.

GIMP(General Image Manipulation Program) is an open source paint program that
can potentially be adapted to many different kinds of image data types and
color models, and can be installed freely on every Unix and Windows machine. 

Recently, GIMP and Rhythm \& Hues' developers ported GIMP 1.0 to floating point
and 16-bit data types. This 16-bit version of GIMP became a test of the
possibility of using GIMP in a film production environment. Although still
missing some important features, this version has proven that GIMP has the
potential of becoming a serious tool for film and high color resolution work. 

With the unstable future of many of the current 16bit paint programs, GIMP is
an attractive option because it is open source and can be customized easily.
It does needs additional development though to be able to fully fill the place
of current production paint choices. 

This document outlines how studios could contribute resources to the GIMP
project to make this possible, and what kind of work is involved with this
effort.  

Section 2 will give a description of the current state of GIMP. Section 3
and 4 will describe future versions of GIMP and its underlying image processing
library, GEGL. Sction 5 describes how GIMP is being currently used in
production at Rhythm and Hues Studios. Section 6 describes a scenario of how
the studios could work together to turn GIMP into a mature production paint
tool.
