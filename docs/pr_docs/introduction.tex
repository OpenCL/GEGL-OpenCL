For special effect houses that deal with images with deep color being able to manipulate frames without color changes is extremely important. It saves the user from having to apply the same color change to all the frames in a scene. 

For a paint program not to produce a color change when manipulating a frame, it is important that it does its operations in the right data type and color space and with the right white point. There is no standard and each company seem to have their own prepriater file format with their own special white point. This makes touching a single image quite expensive due to the extra work that is necessary to change the whole sequence to match the color change. 

The GIMP is an open source paint program that can be tailored to the specific data type, color model, and white point that your company might have. A few GIMP developers together with Rhythm \& Hues' developers ported the GIMP to a floating point paint program. This 16 bit version of the GIMP, although lacking in various aspect, has proven that the GIMP has the potential of becoming a house hold name for painters in the special effect world. 

With the current unstable future of many of the current paint program, the GIMP is an  attractive option because it is open source and therefor can be tailored to each individual need and it has been proven that it can be used in production. It is clear that it needs major enhancement to be able to fill the shoes of the current dying paint programs. This document hopes to convince you that "yes, we want to help in the future development of the GIMP" by giving a clear view of what is needed to make the GIMP into a paint package that can compete with the current major off the shelf paint program. 
Section one will give a description of the current state of the GIMP, section three and four will describe the ....
The last section will describe a possible scenario of how the studios could work to gether to create a mature paint program.


